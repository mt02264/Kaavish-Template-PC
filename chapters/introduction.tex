\section{Problem Statement}
Karachi, the city home to 21.2 million people is amidst a “Garbage Crisis”, a crisis that continues to pollute the city effecting the health and safety of millions of people. It is ranked to be the 134th in a list of 140 cities as the world’s least livable cities. It is Crisis, a problem which requires immediate attention.\\
\\
The garbage crisis encompasses lifting the garbage up as well as disposing it off and not dumping it on any piece of land. The ongoing situation of trash is such that people are dumping trash wherever they find heaps of trash and no adequate measures are being taken in order to clean those spots. The crises seems to have no end in sight as each government has made promise to clean up the economic and the financial hub of the country but none of them have been able to fulfill their commitment due to countless reasons which include lack of funds, inadequate resources and incompetence.\\
\\
Hence, in a attempt to solve a large part of this crisis we'd be addressing the problem of quantification of Garbage to aid the current resource allocation process.


\section{Proposed Solution}

We aim to build an application that is a no cost, crowd sourced solution that address the problem of quantification of garbage in a dump. In order to do this we will be using a two-step approach. The first step is to build a mobile application that allows citizens to upload images of garbage to a server. The second step then involves analysis on the image using image processing techniques in order to quantify the data.\\
\\
Now the process of quantification of data is further divided into two steps. The first step being the identification of the garbage and the second step is the quantification of it.\\
\\
The application will be able to detect and coarsely segment garbage regions in a user-clicked Geo-tagged image. It will then be processing the image, getting the location and estimate of the garbage, resulting in providing an efficient route for a garbage truck in order for it to use as less resources as possible.
A detailed description of each module of the system is presented later in Chapter ~\ref{chap:intro}.

\section{Intended User}

The intended users are the citizens of Karachi, who are tired of seeing dumps of garbage in their neighbourhood and want to contribute towards getting rid of this problem by doing their part that is taking a picture and uploading it. This would result in the authorities not only being notified but also result in them being provided with adequate information for resource allocation.

\section{Key Challenges}

Some of the key challenges faced were:\\
\\
1)Integrating of all the modules.\\
2)Familiarizing ourselves with cloud computing which was a fairly new domain for us to explore.\\
3)Free and limited trials for additional modules.\\
4)Making the flow of the final product as smooth as possible and minimizing the processing delay.\\
5)Working with Android Studio as a framework for the application development.
